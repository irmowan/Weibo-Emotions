% TEX encoding = UTF-8 Unicode
% Complied by LaTeX, not XeLaTeX
% if documents are needed, using BibTeX first after aux file generated.

\documentclass[12pt, oneside]{article}   	
\usepackage{geometry}                		% See geometry.pdf to learn the layout options. There are lots.
\geometry{a4paper}                   		% letterpaper or a4paper or a5paper or ... 
\usepackage{graphicx}				% Use pdf, png, jpg ...
\usepackage{caption}
\usepackage{subcaption}
\usepackage{amssymb}				% Math package
\usepackage{CJKutf8}				% set Chinese support
\usepackage{indentfirst}				% Keep 2 space indent at the begin of a new chapter.
\usepackage{setspace}				% set line space
\usepackage{amsmath}				% Math package, like align
\usepackage{bbm}					% use indicator function in math

\title{微博情绪分析}
\author{
	Yuxin Chen\\
	\texttt{13307130xxx}
	\and
	Yimu Wan\\
	\texttt{13307130xxx}
	\and
	Yong Xu\\
	\texttt{12307130xxx}
	\and
	Shumin Wang\\
	\texttt{12307130xxx}
}
\date{\today}

\begin{document}
\begin{spacing}{1.5}
\begin{CJK}{UTF8}{gbsn}

\maketitle

% \tableofcontents

\newpage

\section{介绍}

\section{结果}
\subsection{微博量-时间}
\label{subsec:weibo_time}
在这一小节中我们统计了微博量与时间的关系。程序实现中,首先将每一条微博 map 为 {\it(HH:MM, emotion)} 的 key-value pair,其中 HH:MM 表示小时和分钟数, emotion 表示情绪(这里复用了\ref{subsec:emotion_time}节中的程序)。再对 map 好的 key-value pair 根据键的值来做 reduce 。 

图\ref{fig:weibo_time}展示了以小时为间隔的统计结果。数据显示,9时-24时微博量相对稳定,高峰出现在22时-23时这一区间内。23时之后微博量开始下降,低谷在5时-6时出现,呈现一个明显的U形,符合大部分人的作息习惯。

\begin{figure}
	\centering
	\includegraphics[width=0.8\linewidth]{../result/charts/weibo_time}
	\caption{微博量-时间}
	\label{fig:weibo_time}
\end{figure}

\subsection{微博情绪-时间}
\label{subsec:emotion_time}
在这一节中我们试图通过统计来揭示微博情绪随时间的变化情况。程序实现中,map和reduce操作在\ref{subsec:weibo_time}节中已经介绍,这里补充说明 {\it(HH:MM, emotion)} 的 key-value pair 中 emotion 的计算方法。计算方法如下:
$$
emotion = 
\begin{cases}
1 & npos > 0~and~nneg = 0 \\
-1 & npos = 0~and~nneg > 0 \\
2 & nneg > 0~and~npos >= nneg \\
-2 & npos > 0~and~nneg > npos \\
0 & \text{otherwise}
\end{cases}
$$
公式中的 {\it npos, npeg} 为预处理后统计出来的两个数值,代表微博中包含的正负表情个数。 公式的结果中,1和-1代表这条微博具有明显的情绪倾向,0代表微博中不包含含有情绪的表情(或者表情不在数据集的情感词典中出现),2和-2代表这条微博的情绪有歧义,但更偏向积极或更偏向消极。

图\ref{fig:emotion_time_1}反映了有明确情绪的微博和总微博数量之间的关系。没有明确情绪的微博由上述 emotion 为 0和±2的微博组成。图\ref{fig:emotion_time_2}反映了积极微博和消极微博数量上的关系,可以明显的看出,情感积极的微博总是占据大多数的。为了更好的体现微博整体情感的变化趋势,我们使用 {\it PN ratio} 来衡量整个微博的情感。
$$PN~ratio = \frac{\text{\#positive weibo}}{\text{\#negative weibo}}$$
图\ref{fig:emotion_time_3}反映了 {\it PN ratio} 与时间的关系。有意思的是,整体的图形与微博量-时间的图形非常相似,这也就意味着,在深夜的时候,消极微博占的比重更大。全天来看,4时开始整体开始往积极方向发展,而到了15时之后,则开始往消极方向发展。

\begin{figure}
	\centering
	\includegraphics[width=0.8\linewidth]{../result/charts/emotion_time_1}
	\caption{有明确微博和总微博数量的关系。图中绿色线表示所有微博量,蓝色为积极情绪,红色为消极情绪。}
	\label{fig:emotion_time_1}
\end{figure}
\begin{figure}
	\centering
	\includegraphics[width=0.8\linewidth]{../result/charts/emotion_time_2}
	\caption{微博情绪-时间(不包括没有明确情绪的微博)。图中蓝色代表积极情绪,红色代表消极情绪,蓝色柱堆叠于红色柱上方。}
	\label{fig:emotion_time_2}
\end{figure}
\begin{figure}
	\centering
	\includegraphics[width=0.8\linewidth]{../result/charts/emotion_time_3}
	\caption{积极与消极微博比例和时间的关系。}
	\label{fig:emotion_time_3}
\end{figure}

\subsection{微博情绪-日期}
这一小节我们分析每日微博情绪(使用 {\it PN ratio} 衡量)波动,并以此检验衡量标准是否有效。从\cite{bollen2011twitter}中作者的实验来看,好的衡量标准应该能够准确的反映重大公众事件的发生。因此我们在实验之前,预期 {\it PN ratio} 在春节等节日应该达到最大,在5·12等纪念日时达到低谷。

实验结果符合我们的预期, {\it PN ratio} 能够如实体现重大公众事件发生后大众心理的变化。积极情绪中:2012年01月01日,元旦,微博情绪达到了全年最高峰, {\it PN ratio} 为5.87。2012年01月22日,除夕,{\it PN ratio} 达到了另一个极大值5.76。随后的大年初一,该值仍维持较高的水平,达到5.51。到了2012年01月28日,春节7天假期结束,上班族不得不重返工作岗位,微博相应的出现了较多的消极情绪,这一天 {\it PN ratio} 为2.96。全年还有两个明显的峰值,一个是在2012年02月14日的情人节,另一个则是在2012年06月01日的六一儿童节。

消极情绪方面:2012年05月12日,汶川大地震四周年,微博上包括名人、机构在内的众多网友纷纷发微博表示悼念, {\it PN ratio} 为2.29。此外,2012年07月22日和2012年07月23日两天达到了全年的最低谷, {\it PN ratio} 分别为2.09和2.19。经过采样分析,主要与两件事情有关,一是2012年07月22日重创北京的大暴雨,二是2011年7月23日发生的甬台温铁路列车追尾事故。两件事情的叠加,使得微博上出现了更多悼念和问责的声音,最终把情绪拉向了全年的最低谷。初次之外,2012年01月28日还出现了一次低谷,{\it PN ratio} 为2.96,经过分析,与惠特尼·休斯顿去世有关。通过数据还可以类似的找出其于公众事件的关联,在此不再赘述。

\begin{figure}
	\centering
	\includegraphics[width=0.8\linewidth]{../result/charts/emotion_day}
	\caption{2012年日微博情绪波动。}
	\label{fig:emotion_day}
\end{figure}

\subsection{微博情绪-性别}
这一小节分析微博情绪在男女之中是否有所差别。从数据中看,积极微博的占比在男女性中不相上下,而女性的消极微博占比要多出约2个百分点。

\begin{figure}
	\centering
	\begin{subfigure}[b]{0.45\linewidth}
		\centering
		\includegraphics[trim = 1.5cm 0 1.5cm 0, clip = true, width=\textwidth]{../result/charts/emotion_gender_m}
		\caption{男性}
	\end{subfigure}
	\begin{subfigure}[b]{0.45\linewidth}
		\centering
		\includegraphics[trim = 1.5cm 0 1.5cm 0, clip = true, width=\textwidth]{../result/charts/emotion_gender_f}
		\caption{女性}
	\end{subfigure}
	\caption{各种微博情绪在不同性别人群里的分布。(图中蓝色代表积极情绪的微博,绿色代表无情绪,橙色代表消极情绪,剩余两类为有歧义的微博)}
	\label{fig:emotion_gender}
\end{figure}

\subsection{微博情绪-身份}
从图\ref{fig:emotion_identification}中蓝色部分(积极的微博)可以看出,最积极的是学校的微博,其次是企业和名人所发。原来预计政府应该在这一项中有最高的比重,但是可以看到,由于政府的微博有很大一部分被划分为无表情一类,推测可能的原因是政府使用的表情更偏向那些没有明显感情色彩的表情。另一方面,从消极占比来看,政府的消极微博占比则相较总体情况要少了很多,由此可以判断政府的微博还是比大部分的用户要更加积极的。除了政府,学校、名人、企业和媒体相对来说负面情绪要少一些,而负面情绪在微博会员这一类别中表现的更强烈。

\begin{figure}
	\centering
	\begin{subfigure}[b]{0.3\linewidth}
		\centering
		\includegraphics[trim = 1.5cm 0 1.5cm 0, clip = true, width=\textwidth]{../result/charts/emotion_identification_all}
		\caption{总体}
	\end{subfigure}
	\begin{subfigure}[b]{0.3\linewidth}
		\centering
		\includegraphics[trim = 1.5cm 0 1.5cm 0, clip = true, width=\textwidth]{../result/charts/emotion_identification_ordinary}
		\caption{普通用户}
	\end{subfigure}
	\begin{subfigure}[b]{0.3\linewidth}
		\centering
		\includegraphics[trim = 1.5cm 0 1.5cm 0, clip = true, width=\textwidth]{../result/charts/emotion_identification_government}
		\caption{政府}
	\end{subfigure}
	\\
	\begin{subfigure}[b]{0.3\linewidth}
		\centering
		\includegraphics[trim = 1.5cm 0 1.5cm 0, clip = true, width=\textwidth]{../result/charts/emotion_identification_campus}
		\caption{学校}
	\end{subfigure}
	\begin{subfigure}[b]{0.3\linewidth}
		\centering
		\includegraphics[trim = 1.5cm 0 1.5cm 0, clip = true, width=\textwidth]{../result/charts/emotion_identification_celebrities}
		\caption{名人}
	\end{subfigure}
	\begin{subfigure}[b]{0.3\linewidth}
		\centering
		\includegraphics[trim = 1.5cm 0 1.5cm 0, clip = true, width=\textwidth]{../result/charts/emotion_identification_media}
		\caption{媒体}
	\end{subfigure}
	\\
	\begin{subfigure}[b]{0.3\linewidth}
		\centering
		\includegraphics[trim = 1.5cm 0 1.5cm 0, clip = true, width=\textwidth]{../result/charts/emotion_identification_enterprises}
		\caption{企业}
	\end{subfigure}
	\begin{subfigure}[b]{0.3\linewidth}
		\centering
		\includegraphics[trim = 1.5cm 0 1.5cm 0, clip = true, width=\textwidth]{../result/charts/emotion_identification_junior}
		\caption{普通会员}
	\end{subfigure}
	\begin{subfigure}[b]{0.3\linewidth}
		\centering
		\includegraphics[trim = 1.5cm 0 1.5cm 0, clip = true, width=\textwidth]{../result/charts/emotion_identification_senior}
		\caption{高级会员}
	\end{subfigure}
	\caption{被转发过的微博的情绪分布。}
	\label{fig:emotion_identification}
\end{figure}


\subsection{微博情绪-转发量}
为了检验什么样的情绪更容易在微博上被传播,我们使用filter过滤出含有转发符号(“//”)的微博。从图\ref{fig:emotion_forward}中可以明显的比较出,积极情绪的微博更容易被转发。

\begin{figure}
	\centering
	\begin{subfigure}[b]{0.45\linewidth}
		\centering
		\includegraphics[trim = 1.5cm 0 1.5cm 0, clip = true, width=\textwidth]{../result/charts/emotion_identification_all}
		\caption{总体}
	\end{subfigure}
	\begin{subfigure}[b]{0.45\linewidth}
		\centering
		\includegraphics[trim = 1.5cm 0 1.5cm 0, clip = true, width=\textwidth]{../result/charts/emotion_forward}
		\caption{被转发过的微博}
	\end{subfigure}
	\caption{被转发过的微博的情绪分布。}
	\label{fig:emotion_forward}
\end{figure}


\subsection{微博情绪-波浪线}
根据经验,人们在发布积极情绪的微博时,更容易加上波浪线(\~{}),比如“么么哒\~{}\~{}”、“好漂亮\~{}\~{}”。为了检验这一经验是否符合实际情况,我们对此进行了统计。如图\ref{fig:emotion_tilde}所示,数据显示,含有波浪线的微博积极情占比从60.3\%提高到了65.6\%,而消极情绪的占比降低了6.4\%,符合我们日常经验。

\begin{figure}
	\centering
	\begin{subfigure}[b]{0.45\linewidth}
		\centering
		\includegraphics[trim = 1.5cm 0 1.5cm 0, clip = true, width=\textwidth]{../result/charts/emotion_identification_all}
		\caption{总体}
	\end{subfigure}
	\begin{subfigure}[b]{0.45\linewidth}
		\centering
		\includegraphics[trim = 1.5cm 0 1.5cm 0, clip = true, width=\textwidth]{../result/charts/emotion_tilde}
		\caption{含有波浪线的微博}
	\end{subfigure}
	\caption{含有波浪线的微博情绪分布情况。}
	\label{fig:emotion_tilde}
\end{figure}

\subsection{微博情绪词典}
\label{subsec:polar_dict}
我们可以首先利用表情和表情畸形词典统计得到的每条微博的情绪,来生成一个描述单词的极性词典,具体做法如下:首先进行分词,这里使用了结巴开源库;之后计算每个词出现的次数和出现在积极微博中的次数,将两者相除,就可以得到一个 $[0, 1]$ 之间的值来衡量这个词的极性。
$$\text{词word的积极程度}= \frac{\text{word出现在积极微博中的次数}}{\text{word出现在所有微博中的次数}}$$
在这里,我们定义如果一个词大于0.6,那么它就是积极的,介于0.6和0.4之间的是无明显情绪的,小于0.4的是消极的。
表\ref{tbl:polar_dict}列出了最后得到的一些结果。

\begin{table}[]
\centering
\begin{tabular}{|c|c|c|c|c|c|}
\hline
\multicolumn{2}{|c|}{积极} & \multicolumn{2}{c|}{消极} & \multicolumn{2}{c|}{中性} \\ \hline
词           & 积极度        & 词          & 积极度        & 词          & 积极度        \\ \hline
新婚          & 0.87       & 他*的        & 0.20       & 楼盘         & 0.50       \\ \hline
爱           & 0.89       & 艹          & 0.24       & 毕业生        & 0.50       \\ \hline
天天开心        & 0.89       & 骗          & 0.39       & 高管         & 0.48       \\ \hline
非常感谢        & 0.86       & MLGB       & 0.34       & 美国         & 0.53       \\ \hline
好样          & 0.85       & 睡眠不足       & 0.29       & 组图         & 0.50       \\ \hline
情人节         & 0.71       & 哭泣         & 0.34       & 研究所        & 0.58       \\ \hline
\end{tabular}
\caption{生成的微博情绪词典样例。}
\label{tbl:polar_dict}
\end{table}

\subsection{表情的积极程度}
数据集提供的表情仅用三种值来划分它的极性(积极、消极和无情绪),在此,我们进行了一些统计,得到了每个表情的积极程度。具体做法如下:首先对微博进行分词,分词方法同\ref{subsec:polar_dict}一节;之后使用\ref{subsec:polar_dict}一节得到的极性词典,来计算每一条微博的积极程度,用 $[-1, 1]$ 之间的实数来表示;最后,对每个表情计算其积极程度。
使用下式来计算每一条微博的情绪:
$$\text{微博的情绪} = \frac{\sum{\text{所有出现过的单词的积极程度}}}{\text{\#所有出现过的单词}}$$
使用下式来计算每个表情的积极程度:
$$\text{表情的积极程度} = \frac{\sum{\text{包含该单词的微博的情绪}}}{\text{\#包含该单词的微博}}$$

表\ref{tbl:weighted_expreesions}展示了一部分结果。图\ref{fig:weighted_expressions_top5}是最积极和最消极的表情。

\begin{figure}
	\centering
	\includegraphics[width=0.35\linewidth]{../result/charts/weighted_expressions_top5}
	\caption{积极程度最高和最低的5个表情,第一行为最积极的表情,第二行为最消极的表情,从左至右程度依次递减。}
	\label{fig:weighted_expressions_top5}
\end{figure}

\begin{table}[]
\centering
\begin{tabular}{|c|c|c|c|}
\hline
\multicolumn{2}{|c|}{积极} & \multicolumn{2}{c|}{消极} \\ \hline
表情            & 积极程度     & 表情          & 积极程度      \\ \hline
{[}太开心{]}     & 0.887    & {[}白眼{]}    & -0.769    \\ \hline
{[}圣诞袜{]}     & 0.905    & {[}哼{]}     & -0.748    \\ \hline
{[}给力{]}      & 0.924    & {[}怒{]}     & -0.855    \\ \hline
{[}来{]}       & 0.853    & {[}悲伤{]}    & -0.785    \\ \hline
{[}礼物{]}      & 0.879    & {[}鄙视{]}    & -0.782    \\ \hline
{[}可爱{]}      & 0.898    & {[}委屈{]}    & -0.710    \\ \hline
\end{tabular}
\caption{表情的积极程度样例}
\label{tbl:weighted_expreesions}
\end{table}

\newpage
\renewcommand\refname{参考文献}
\bibliographystyle{plain}
\bibliography{Report}

\end{CJK}
\end{spacing}
\end{document}
